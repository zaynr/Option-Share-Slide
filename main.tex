\documentclass[12pt]{ctexbeamer}	%声明文档类型为beamer幻灯片
\mode <presentation>

%\usetheme{default}
%\usetheme{AnnArbor}
%\usetheme{Antibes}
%\usetheme{Bergen}
%\usetheme{Berkeley}
%\usetheme{Berlin}
%\usetheme{Boadilla}
%\usetheme{CambridgeUS}
%\usetheme{Copenhagen}
% \usetheme{Darmstadt}
%\usetheme{Dresden}
%\usetheme{Frankfurt}
%\usetheme{Goettingen}
%\usetheme{Hannover}
% \usetheme{Ilmenau}
% \usetheme{JuanLesPins}
% \usetheme{Luebeck}
\usetheme{Madrid}
% \usetheme{Malmoe}
%\usetheme{Marburg}
%\usetheme{Montpellier}
%\usetheme{PaloAlto}
%\usetheme{Pittsburgh}
%\usetheme{Rochester}
%\usetheme{Singapore}
%\usetheme{Szeged}
%\usetheme{Warsaw}

% \usecolortheme{albatross}
\usecolortheme{beaver}
% \usecolortheme{beetle}
% \usecolortheme{crane}
% \usecolortheme{dolphin}
% \usecolortheme{dove}
% \usecolortheme{fly}
% \usecolortheme{lily}
% \usecolortheme{orchid}
% \usecolortheme{rose}
% \usecolortheme{seagull}
% \usecolortheme{seahorse}
% \usecolortheme{whale}
% \usecolortheme{wolverine}

%\beamersetaveragebackground{black!10}

\setbeamercovered{transparent}
\setbeamertemplate{navigation symbols}{}

\usefonttheme{professionalfonts}
\useinnertheme{circles}%{rectangles}
\setbeamertemplate{itemize item}{$\circledast$}%{\checkmark}


\setbeamertemplate{title page}
{
%  \vbox{}
  \vfill
  \begin{centering}
  \includegraphics[height=2.0cm]{figure/hundsun-logo.png}
        \vskip0.6em\par%
    \begin{beamercolorbox}[sep=8pt,center,shadow=true,rounded=true]{title}
      \usebeamerfont{title}\inserttitle\par%
      \ifx\insertsubtitle\@empty%
      \else%
        \vskip0.25em%
        {\usebeamerfont{subtitle}\usebeamercolor[fg]{subtitle}\insertsubtitle\par}%
      \fi%
    \end{beamercolorbox}%
    \vskip1em\par
    \begin{beamercolorbox}[sep=8pt,center]{author}
      \usebeamerfont{author}\insertauthor
    \end{beamercolorbox}
    \begin{beamercolorbox}[sep=8pt,center]{institute}
      \usebeamerfont{institute}\insertinstitute
    \end{beamercolorbox}
    \begin{beamercolorbox}[sep=8pt,center]{date}
      \usebeamerfont{date}\insertdate
    \end{beamercolorbox}\vskip0.5em
  \end{centering}
  \vfill
}

% \hypersetup{pdfpagemode=FullScreen} % makes your presentation go automatically to full screen

\usepackage{fontspec,xunicode,xltxtra}

\usepackage{multimedia}	%让文档支持多媒体
\usepackage{graphics}	%让文档支持图片
\usepackage{hyperref}	%让文档支持超链接
\usepackage{booktabs}	%让文档支持三线表格
\usepackage{amsmath}	%ams可以让文档支持数学公式
\usepackage{amsfonts}
\usepackage{amssymb}
\usepackage{color}
\usepackage{graphicx,psfrag}
\usepackage{epsfig}

\title[期权知识分享]{关于期权的基础知识及套利过程}	%幻灯片标题
\author[Zengyu Zeng]{曾锃煜}

\institute[Hundsun]{资管,研发四部}	%作者单位
\date{\today}

\setcounter{tocdepth}{1}

\begin{document}
\begin{frame}
\titlepage
\end{frame}

\AtBeginSection[]{
  \frame{
    \frametitle{大纲}
    \tableofcontents[currentsection]
  }
}

\begin{frame}
\frametitle{大纲}
\tableofcontents
\end{frame}

%%%%%%%%%%%%%%%%%%%%%%%%%%%%%%%%%%%%%%%%%%%%%%%%%%%%%%%%%%%%%%%%%%%%%%%%%%%%%%%

\section{基本概念}

\begin{frame}{期权定义}
  \begin{itemize} %enumerate;description
    \item 约定价格(对赌)
    \item 权利金
    \item 行使权利
    \item 到期日、行权日、行权时间
  \end{itemize}
\end{frame}

\begin{frame}{期权分类}
  \begin{block}{买卖方向}
    \begin{itemize}
      \item 认购期权
      \item 认沽期权
    \end{itemize}
  \end{block}
  \begin{block}{期权内涵价值}
    \begin{itemize}
      \item 实值期权
      \item 虚值期权
      \item 平值期权
    \end{itemize}
  \end{block}
\end{frame}

\begin{frame}{中国期权交易}
  \begin{block}{投资门槛}
    \begin{itemize}
      \item 托管与资产不少于 100 万 RMB
      \item 通过测试
      \item 有模拟经历
      \item 信用
    \end{itemize}
  \end{block}
\end{frame}

\begin{frame}{中国期权交易}
  \begin{block}{衍生品合约账户}
    \begin{itemize}
      \item 记录合约持仓
      \item 期权交易申报
      \item 行权申报
    \end{itemize}
    不可用于存放现货证券,行权的交割标的物通过普通证券账户完成。
  \end{block}
  \begin{block}{衍生品保证金账户}
    \begin{itemize}
      \item 权利金
      \item 卖方保证金
      \item 行权资金交收
    \end{itemize}
  \end{block}
\end{frame}

\begin{frame}{中国期权交易}
  国内目前有 ETF 期权与个股期权进行交易。此处以上证 50ETF 作为期权交易举例。标的即为 50ETF,Exchange Trading Fund。
  \begin{block}{上证 50 指数}
    \begin{enumerate}
      \item 50
      \item 加权
      \item 算法:$$\frac{a \times b}{c}$$a = 实时加权股票净值, b = 基准指数, c = 基准日加权股票净值
    \end{enumerate}
  \end{block}
\end{frame}

\begin{frame}{中国期权交易}
  \begin{block}{50ETF}
    \begin{itemize}
      \item 指数基金
      \item 上证 50 指数
      \item 交易成本
    \end{itemize}
  \end{block}
  \begin{block}{交易类型}
    有六个类型,分别是“买入”、“卖出”、“备兑”的开仓与平仓。
  \end{block}
\end{frame}

%%%%%%%%%%%%%%%%%%%%%%%%%%%%%%%%%%%%%%%%%%%%%%%%%%%%%%%%%%%%%%%%%%%%%%%%%%%%%%%

\section{套利}

\begin{frame}{第一步}
\end{frame}

%%%%%%%%%%%%%%%%%%%%%%%%%%%%%%%%%%%%%%%%%%%%%%%%%%%%%%%%%%%%%%%%%%%%%%%%%%%%%%%

\begin{frame}
\centerline{\Large 谢谢!}
\end{frame}

\end{document}
