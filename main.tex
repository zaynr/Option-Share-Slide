\documentclass[12pt]{ctexbeamer}	%声明文档类型为beamer幻灯片
\mode <presentation>

%\usetheme{default}
%\usetheme{AnnArbor}
%\usetheme{Antibes}
%\usetheme{Bergen}
%\usetheme{Berkeley}
%\usetheme{Berlin}
%\usetheme{Boadilla}
%\usetheme{CambridgeUS}
%\usetheme{Copenhagen}
% \usetheme{Darmstadt}
%\usetheme{Dresden}
%\usetheme{Frankfurt}
%\usetheme{Goettingen}
%\usetheme{Hannover}
% \usetheme{Ilmenau}
% \usetheme{JuanLesPins}
% \usetheme{Luebeck}
\usetheme{Madrid}
% \usetheme{Malmoe}
%\usetheme{Marburg}
%\usetheme{Montpellier}
%\usetheme{PaloAlto}
%\usetheme{Pittsburgh}
%\usetheme{Rochester}
%\usetheme{Singapore}
%\usetheme{Szeged}
%\usetheme{Warsaw}

% \usecolortheme{albatross}
\usecolortheme{beaver}
% \usecolortheme{beetle}
% \usecolortheme{crane}
% \usecolortheme{dolphin}
% \usecolortheme{dove}
% \usecolortheme{fly}
% \usecolortheme{lily}
% \usecolortheme{orchid}
% \usecolortheme{rose}
% \usecolortheme{seagull}
% \usecolortheme{seahorse}
% \usecolortheme{whale}
% \usecolortheme{wolverine}

%\beamersetaveragebackground{black!10}

\setbeamercovered{transparent}
%\setbeamertemplate{navigation symbols}{}

\usefonttheme{professionalfonts}
\useinnertheme{circles}%{rectangles}
\setbeamertemplate{itemize item}{$\circledast$}%{\checkmark}


\setbeamertemplate{title page}
{
%  \vbox{}
  \vfill
  \begin{centering}
%   \includegraphics[height=2.0cm]{figures/ustc_logo_fig.pdf}
        \vskip0.6em\par%
    \begin{beamercolorbox}[sep=8pt,center,shadow=true,rounded=true]{title}
      \usebeamerfont{title}\inserttitle\par%
      \ifx\insertsubtitle\@empty%
      \else%
        \vskip0.25em%
        {\usebeamerfont{subtitle}\usebeamercolor[fg]{subtitle}\insertsubtitle\par}%
      \fi%
    \end{beamercolorbox}%
    \vskip1em\par
    \begin{beamercolorbox}[sep=8pt,center]{author}
      \usebeamerfont{author}
      {
            \insertauthor\\
      }
    \end{beamercolorbox}
    \begin{beamercolorbox}[sep=8pt,center]{institute}
      \usebeamerfont{institute}\insertinstitute
    \end{beamercolorbox}
    \begin{beamercolorbox}[sep=8pt,center]{date}
      \usebeamerfont{date}\insertdate
    \end{beamercolorbox}\vskip0.5em
  \end{centering}
  \vfill
}

% \hypersetup{pdfpagemode=FullScreen} % makes your presentation go automatically to full screen

\usepackage{fontspec,xunicode,xltxtra}

\usepackage{multimedia}	%让文档支持多媒体
\usepackage{graphics}	%让文档支持图片
\usepackage{hyperref}	%让文档支持超链接
\usepackage{booktabs}	%让文档支持三线表格
\usepackage{amsmath}	%ams可以让文档支持数学公式
\usepackage{amsfonts}
\usepackage{amssymb}
\usepackage{color}
\usepackage{graphicx,psfrag}
\usepackage{epsfig}

\title[期权知识分享]{关于期权的基础知识及套利过程}	%幻灯片标题
\author[Zengyu Zeng]{曾锃煜}

\institute[Hundsun]{资管研发四部}	%作者单位
\date{\today}

\setcounter{tocdepth}{1}

\begin{document}
\begin{frame}
\titlepage
\end{frame}

\AtBeginSection[]{
  \frame{
    \frametitle{大纲}
    \tableofcontents[currentsection]
  }
}

\begin{frame}
\frametitle{大纲}
\tableofcontents
\end{frame}

%%%%%%%%%%%%%%%%%%%%%%%%%%%%%%%%%%%%%%%%%%%%%%%%%%%%%%%%%%%%%%%%%%%%%%%%%%%%%%%

\section{概述}

\begin{frame}{研究背景}
  \begin{block}{传统关系抽取}
    \begin{itemize}
      \item 优点:提取关系基于匹配规则,速度快
      \item 缺点:规则需要预先定义,适用范围窄
    \end{itemize}
  \end{block}
  \pause
  \begin{block}{开放式关系抽取}
    \begin{itemize}
      \item 优点:应用范围广,只需要语料库作为输入数据即可进行处理
      \item 缺点:初始信息少、语义类别难确定、依赖于训练语料
    \end{itemize}
  \end{block}
\end{frame}

%%%%%%%%%%%%%%%%%%%%%%%%%%%%%%%%%%%%%%%%%%%%%%%%%%%%%%%%%%%%%%%%%%%%%%%%%%%%%%%

\section{中文开放式关系抽取实现方式}

\begin{frame}{分词与依存句法分析}
%   \epsfig{figure=figures/dependency.PNG,width=\textwidth}

\end{frame}

\begin{frame}{提取候选关系}
%   \epsfig{figure=figures/sentence1.PNG,width=\textwidth}
\end{frame}

%%%%%%%%%%%%%%%%%%%%%%%%%%%%%%%%%%%%%%%%%%%%%%%%%%%%%%%%%%%%%%%%%%%%%%%%%%%%%%%

\section{实验}

\begin{frame}{评估标准}
  \begin{description}
    \item[精确率]<1-> $$Precision=\frac{|T|}{|A|}\times 100\%$$
    \item[召回率]<2-> $$Recall=\frac{|T|}{|S|}\times 100\%$$
    \item[F1 值]<3-> $$F1=\frac{2\times Precision\times Recall}{Precision+Recall}$$
  \end{description}
\end{frame}

\begin{frame}{评估准备}
%   \epsfig{figure=figures/result.PNG,width=\textwidth}
  \begin{block}{人工标注关系文件}
    \begin{itemize}
      \item 中文文本语句内的关系数量
      \item 谓词短语
      \item 论元内容
      \item 论元数目
    \end{itemize}
  \end{block}
\end{frame}

\begin{frame}{评估程序}
    \begin{enumerate}
      \item 以语义模式重复频数阈值 $t^{sem}$ 作为自变量,运行中文开放式关系抽取系统处理语料库
      \item 按语句将机器提取的关系与人工标注的关系进行比对并统计、输出数据
    \end{enumerate}
\end{frame}

%%%%%%%%%%%%%%%%%%%%%%%%%%%%%%%%%%%%%%%%%%%%%%%%%%%%%%%%%%%%%%%%%%%%%%%%%%%%%%%

\section{结果分析}

\begin{frame}{ZORE 基于“燃规3”语料库的运行结果}
  \begin{columns}
    \begin{column}{0.70\textwidth}
    %   \epsfig{figure=figures/performance.PNG,width=\textwidth}
    \end{column}
    \begin{column}{0.30\textwidth}
      \begin{block}{结论}
        体现了先前提及的对召回率 Recall 与精确度 Precision 的平衡性权衡,F1 值最大处关系抽取系统性能最好。
      \end{block}
    \end{column}
  \end{columns}
\end{frame}

\begin{frame}{ZORE 与 DPM 基于“燃规3”语料库的运行结果对比}
  \begin{columns}
    \begin{column}{0.70\textwidth}
    %   \epsfig{figure=figures/compare.PNG,width=\textwidth}
    \end{column}
    \begin{column}{0.30\textwidth}
      \begin{block}{结论}
        推断是是由于该语料库是法规,文本结构较为碎片化,以上下文环境进行分析的兼类词处理难以顺利工作导致的。
      \end{block}
    \end{column}
  \end{columns}
\end{frame}

\begin{frame}{ZORE 与 DPM 基于“燃规3”语料库的运行结果对比}
  \begin{columns}
    \begin{column}{0.70\textwidth}
    %   \epsfig{figure=figures/comparef.PNG,width=\textwidth}
    \end{column}
    \begin{column}{0.30\textwidth}
      \begin{block}{结论}
        可以推断缺少对兼类词处理不是 ZORE 算法提升性能所遇到的瓶颈。
      \end{block}
    \end{column}
  \end{columns}
\end{frame}

%%%%%%%%%%%%%%%%%%%%%%%%%%%%%%%%%%%%%%%%%%%%%%%%%%%%%%%%%%%%%%%%%%%%%%%%%%%%%%%

\section{错误分析}

\begin{frame}{错误分析}
  \begin{enumerate}
    \item $t^{sem}$ 语义模式最低频率限制导致遗漏正确关系
    \item 自然语言处理带来的错误,如将论元识别为“、构筑物”而不是“建、构筑物”
    \item 关系模式选择错误,导致抽取出错误的关系
  \end{enumerate}
\end{frame}

%%%%%%%%%%%%%%%%%%%%%%%%%%%%%%%%%%%%%%%%%%%%%%%%%%%%%%%%%%%%%%%%%%%%%%%%%%%%%%%


\begin{frame}
\centerline{\Large 谢谢!}
\end{frame}

\end{document}
